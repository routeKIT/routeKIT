% das Papierformat zuerst
\documentclass[a4paper, 11pt]{article}
\usepackage[margin=3cm]{geometry}
\usepackage[utf8]{inputenc}
\usepackage[T1]{fontenc}
\usepackage[ngerman]{babel}
\usepackage{hyperref} % clickable refs
\usepackage{graphicx}
\usepackage[toc, numberedsection]{glossaries}
\usepackage{float}

\makeglossary

%Hack for referencing labels
\makeatletter
\def\namedlabel#1#2{\begingroup
    #2%
    \def\@currentlabel{#2}%
    \phantomsection\label{#1}\endgroup
}
\makeatother
% End: Hack for referencing labels

\newcommand{\keine}[0]{
  % see http://tex.stackexchange.com/questions/61155/linespacing-without-packages
  \setlength\lineskiplimit{-1000pt} % allow overflowing of lines
  \linespread{0}\selectfont % don’t advance line after empty item
  \item[]
}
% Argumente:
% 1. Klassenname
% 2. Beschreibung
% 3. Attribute (ein \item[name] pro Attribut, oder \keine)
% 4. Methoden (ein \item[name] pro Methode, oder \keine)
\newcommand{\klasse}[4]{
  \item[\namedlabel{K:#1}{#1}] #2
    \begin{description}
      \item[Attribute] \hfill % \hfill forces the following description into the next line
        \begin{description}
          #3
        \end{description}
      \item[Methoden] \hfill % \hfill forces the following description into the next line
        \begin{description}
          #4
        \end{description}
    \end{description}
}

% Glossar: alle Einträge, aber ohne extra Referenzen
% http://tex.stackexchange.com/questions/115635/glossaries-suppress-pages-when-using-glsaddall
\newcommand*{\glsgobblenumber}[1]{}
\makeatletter
\newcommand*{\glsaddnp}[2][]{
  \glsdoifexists{#2}{
    \def\@glsnumberformat{glsgobblenumber}
    \edef\@gls@counter{\csname glo@#2@counter\endcsname}
    \setkeys{glossadd}{#1}
    \@gls@saveentrycounter
    \@do@wrglossary{#2}
  }
}
\newcommand{\glsaddallunused}[1][]{
  \edef\@glo@type{\@glo@types}
  \setkeys{glossadd}{#1}
  \forallglsentries[\@glo@type]{\@glo@entry}{
    \ifglsused{\@glo@entry}{}{
      \glsaddnp[#1]{\@glo@entry}}}
}
\makeatother

\renewcommand{\glsnamefont}[1]{\mdseries #1} % glossary entries shouldn’t be bold

% Glossar

% So sieht ein Glossar-Eintrag aus:
%
%\newglossaryentry{dijkstra}{
%  name={Dijkstra’s Algorithmus},
%  description={ein Algorithmus, um den optimalen Pfad in einem gerichteten Graphen zu finden}
%}
%\newglossaryentry{arc}{
%  name={Arc-Flags},
%  description={eine Technik, um Routenberechnung zu beschleunigen},
%  see={dijkstra}
%}
%
% Und so kann er im Dokument verwendet werden:
%
% lorem ipsum dolor sit \gls{arc}, consectetur
%
% End: Glossar

% usage: \counteditem{prefix}{refName} -> item `/prefixXX/` with label `prefix:refName` (where XX is counted in increments of 10)
\makeatletter

\begin{document}
\shorthandoff{"}

% place a symbol before clickable links
% this has to come *after* \begin{document} because hyperref installs a \AtBeginDocument hook that updates the ref command.
\newcommand{\refsymbol}[0]{\scalebox{0.5}{$\nearrow$}}
\let\oldref\ref
\renewcommand{\ref}[1]{\refsymbol\oldref{#1}}
\let\oldgls\gls
\renewcommand{\gls}[1]{\refsymbol\oldgls{#1}}
\let\oldGls\Gls
\renewcommand{\Gls}[1]{\refsymbol\oldGls{#1}}
\let\oldglspl\glspl
\renewcommand{\glspl}[1]{\refsymbol\oldglspl{#1}}
\let\oldGlspl\Glspl
\renewcommand{\Glspl}[1]{\refsymbol\oldGlspl{#1}}
\let\oldglslink\glslink
\renewcommand{\glslink}[2]{\refsymbol\oldglslink{#1}{#2}}
\let\oldhyperref\hyperref
\renewcommand{\hyperref}[2][notActuallyOptional]{\refsymbol\oldhyperref[#1]{#2}}
\let\oldautoref\autoref
\renewcommand{\autoref}[1]{\refsymbol\oldautoref{#1}}

\newcommand{\abbildung}[1]{\autoref{fig:#1}}
\newcommand{\routeKIT}[0]{\textit{routeKIT} }

\begin{titlepage}
\makeatletter
\begin{center}
~\\[5em]
{\Huge routeKIT}\\[3em]
{\huge Entwurf}\\[1em]
{\large\today}\\[2.5em]
{\LARGE
Kevin Birke\\
Felix Dörre\\
Fabian Hafner\\
Lucas Werkmeister\\
Dominic Ziegler\\
Anastasia Zinkina\\[3em]}
betreut durch\\[2em]
{\Large
Julian~Arz\\
G.~Veit~Batz\\
Dr.~Dennis~Luxen\\
Dennis~Schieferdecker\\[1em]}
am\\[1em]
{\Large
Karlsruher Institut für~Technologie\\
Institut für Theoretische~Informatik\\
Algorithmik~II\\
\large
Prof.~Dr.~Peter~Sanders}
\end{center}
\makeatother
\end{titlepage}
\tableofcontents
\newpage

\section{Klassen}
\begin{description}

\klasse{Profile}
       {Ein Fahrzeugprofil.}
       {
       \item[name] Der Name des Profils.
       \item[vehicleType] Der Fahrzeugtyp.
       \item[height] Die Höhe des Fahrzeugs, in Zentimetern.
       \item[width] Die Breite des Fahrzeugs, in Zentimetern.
       \item[weight] Das Gewicht des Fahrzeugs, in Kilogramm.
       \item[speedHighway] Die Durchschnittsgeschwindigkeit des Fahrzeugs auf der Autobahn, in Kilometern pro Stunde.
       \item[speedRoad] Die Durchschnittsgeschwindigkeit des Fahrzeugs auf der Landstraße, in Kilometern pro Stunde.
       }
       {\keine}
\klasse{Map}
       {Eine Karte.}
       {
       \item[name] Der Name der Karte.
       \item[renderingData] Die Daten, die zum Rendern der Karte benötigt werden.
       }
       {\keine}
\klasse{ProfileAndMap}
       {Eine Kombination aus einem \hyperref[K:Profile]{Profil} und einer \hyperref[K:Map]{Karte}. Siehe auch \ref{K:MapData}.}
       {
       \item[profile] Das Profil
       \item[map] Die Karte
       }
       {\keine}
\klasse{MapData}
       {Kapselt ein Profil, eine Karte, und die Ergebnisse einer Vorberechnung für diese Profil-Karte-Kombination.}
       {\keine}
       {\keine} % TODO

\klasse{History}
       {Kapselt den Verlauf.}
       {\keine}
       {
       \item[addEntry] Fügt einen Eintrag zum Verlauf hinzu. Als \texttt{date} des neuen Eintrags wird die aktuelle Zeit verwendet.
       \item[getEntries] Gibt alle Einträge des Verlaufs zurück.
       \item[save] Speichert den Verlauf in die angegebene Datei.
       \item[load] Lädt einen Verlauf aus der angegebenen Datei und gibt ihn zurück.
       }
\klasse{HistoryEntry}
       {Ein Eintrag im Verlauf.}
       {
       \item[start] Der Startpunkt der Anfrage.
       \item[destination] Der Zielpunkt der Anfrage.
       \item[date] Der Zeitpunkt der Anfrage.
       }
       {\keine}

\end{description}

\glsaddallunused
\makeatletter
\newglossarystyle{myAltlist}{
  \glossarystyle{altlist} % base this style on altlist
  \renewcommand*{\glossaryentryfield}[5]{
  \item[\glsentryitem{##1}\glstarget{##1}{##2}]
    \mbox{}\par\nobreak\@afterheading
    ##3\glspostdescription\space Auf Seite ##5.
  }
}
\makeatother
\printglossary[type=main, title={Glossar}, toctitle={Glossar}, style=myAltlist]

\end{document}
