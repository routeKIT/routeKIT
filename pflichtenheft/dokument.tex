% das Papierformat zuerst
\documentclass[a4paper, 11pt]{article}
\usepackage[utf8]{inputenc}
\usepackage[T1]{fontenc}
\usepackage[ngerman]{babel}

 
\title{Pflichtenheft}
%\author{}
%\date{}

% hier beginnt das Dokument
\begin{document}
\shorthandoff{"}

\maketitle
\newpage
\tableofcontents
\newpage


\section{Einleitung}
%Lucas
\section{Zielbestimmung}
%Lucas
\subsection{Musskriterien}
\begin{itemize}
\item "Ausreichend komfortables GUI"
\begin{itemize}
\item Rechtsklick auf die Karte, "hier Start", "hier Ziel"
\item Karte anzeigen
\item Karte ziehen/zoomen
\item Weg anfragen
\item Anzeigen der Route
\end{itemize}
\item Berechnen einer Route zwischen 2 Punkten auf der Karte.
\end{itemize}
\subsection{Wunschkriterien}
\begin{itemize}
\item Umsetzen möglichst vieler Beschränkungen (Staßentyp, Abbiegebeschränkung, 
Maximalgewicht, Einbahnstraße) trotz Beschleunigung. Profile für die Vorberechnung.

\item Wegbeschreibung anzeigen. (Kreisverkehre, Zählen)
\item Berechnungsanfragen speichern. History/Favoriten
\item Export, Drucken von Route.

\item Suchen von Adressen (+ POIs)

\item Zwischenziele, (mit DnD oder manuell)
\item Optional OSM-Kachlen
\item Tankstellen/Hotels etc. POIs an der Strecke.
\end{itemize}
\subsection{Abgerenzungskriterien}

\begin{itemize}
\item Kein Routenplaner für Fahrrad, Fußgänger, Bahnfahrer oder etc.
\item Keine Moblie App.
\item Keine Webanwendung
\item Keine Alternativroute sofort anzeigen

\end{itemize}

\section{Produkteinsatz}
%Kevin
\section{Produktumgebung}
%Kevin
\section{Funktionale Anforderungen}
% Felix
\begin{itemize}
\item Berechnen des nächsten Punkts zu Geokoordinaten auf der Karte.
\item 
\end{itemize}
\section{Produktdaten}
%Anastasia
\section{Systemmodell}
%Anastasia
\section{Nichtfunktionale Anforderungen}

\begin{enumerate}
\item Suchen der Route schneller als 5 Sekunden. (  mit Arcflags)

\end{enumerate}
% Dominic
\section{Benutzeroberfläche}
% Fabian
\section{Qualitätsziele}
% Dominic
\section{Globale Testfälle und Szenarien}
% Fabian

\section{Entwicklungsumgebung}
% Kevin
\section{Glossar}

\end{document}
