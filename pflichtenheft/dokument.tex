% das Papierformat zuerst
\documentclass[a4paper, 11pt]{article}
\usepackage[margin=3cm]{geometry}
\usepackage[utf8]{inputenc}
\usepackage[T1]{fontenc}
\usepackage[ngerman]{babel}
\usepackage{hyperref} % clickable refs
\usepackage{graphicx}
\usepackage[toc, numberedsection, style=altlist]{glossaries}
\makeglossary

 
\title{Pflichtenheft}
\author{Kevin Birke \\ Felix Dörre \\ Fabian Hafner \\ Lucas Werkmeister \\ Dominic Ziegler \\ Anastasia Zinkina}
%\date{}

%Hack for referencing labels
\makeatletter
\def\namedlabel#1#2{\begingroup
    #2%
    \def\@currentlabel{#2}%
    \phantomsection\label{#1}\endgroup
}
\makeatother
% End: Hack for referencing labels

% Glossar: alle Einträge, aber ohne extra Referenzen
% http://tex.stackexchange.com/questions/115635/glossaries-suppress-pages-when-using-glsaddall
\newcommand*{\glsgobblenumber}[1]{}
\makeatletter
\newcommand*{\glsaddnp}[2][]{
  \glsdoifexists{#2}{
    \def\@glsnumberformat{glsgobblenumber}
    \edef\@gls@counter{\csname glo@#2@counter\endcsname}
    \setkeys{glossadd}{#1}
    \@gls@saveentrycounter
    \@do@wrglossary{#2}
  }
}
\newcommand{\glsaddallunused}[1][]{
  \edef\@glo@type{\@glo@types}
  \setkeys{glossadd}{#1}
  \forallglsentries[\@glo@type]{\@glo@entry}{
    \ifglsused{\@glo@entry}{}{
      \glsaddnp[#1]{\@glo@entry}}}
}
\makeatother

\renewcommand{\glsnamefont}[1]{\mdseries #1} % glossary entries shouldn’t be bold

% Glossar

% So sieht ein Glossar-Eintrag aus:
%
%\newglossaryentry{dijkstra}{
%  name={Dijkstra’s Algorithmus},
%  description={ein Algorithmus, um den optimalen Pfad in einem gerichteten Graphen zu finden}
%}
%\newglossaryentry{arc}{
%  name={Arc-Flags},
%  description={eine Technik, um Routenberechnung zu beschleunigen},
%  see={dijkstra}
%}
%
% Und so kann er im Dokument verwendet werden:
%
% lorem ipsum dolor sit \gls{arc}, consectetur
%
% End: Glossar

% usage: \counteditem{prefix}{refName} -> item '/prefixXX/' with label 'prefix:refName' (where XX is counted in increments of 10)
\makeatletter
\newcommand{\oitem}[2]{
  % define the counter
  \@ifundefined{c@oitem#1}{\newcounter{oitem#1}}{} % initialized at 0
  \addtocounter{oitem#1}{10}
  \item[\namedlabel{#1:#2}{/#1\arabic{oitem#1}/}]
}
\makeatother

\newcommand{\testfall}[3]{
  \begin{description}
    \item[Vorbedingungen] #1
    \item[Ablauf] #2
    \item[Erwartetes Ergebnis] #3
  \end{description}
}

% Betreuer: Rand reduzieren. (gemacht)
% Betreuer: Wichtig, viel Kriterien, GUI, Testfälle
% Betreuer: Präzise!!!
% hier beginnt das Dokument
\begin{document}
\shorthandoff{"}

% place a symbol before clickable links
% this has to come *after* \begin{document} because hyperref installs a \AtBeginDocument hook that updates the ref command.
\newcommand{\refsymbol}[0]{\scalebox{0.5}{$\nearrow$}}
\let\oldref\ref
\renewcommand{\ref}[1]{\refsymbol\oldref{#1}}
\let\oldgls\gls
\renewcommand{\gls}[1]{\refsymbol\oldgls{#1}}
\let\oldglslink\glslink
\renewcommand{\glslink}[2]{\refsymbol\oldglslink{#1}{#2}}
\let\oldhyperref\hyperref
\renewcommand{\hyperref}[2][notActuallyOptional]{\refsymbol\oldhyperref[#1]{#2}}


\maketitle
\newpage
\tableofcontents
\newpage
% Betreuer: Berechnung von Punkt auf Kante zu Punkt auf Kante.  (KdBaum über Kanten) (gemackt)
% Betreuer: Profile exakt welcher Typ

% alle Glossareintraege
\newacronym[longplural=Personenkraftwagen]{pkw}{PKW}{Personenkraftwagen}
\newacronym[longplural=Lastkraftwagen]{lkw}{LKW}{Lastkraftwagen}
\newacronym{osm}{OSM}{OpenStreetMap}
\newacronym{gui}{GUI}{Graphical User Interface}
\newacronym[longplural={Points of interest}]{poi}{POI}{Point of interest}

\newglossaryentry{profil}{
  name={Profil},
  description={Enthält für die Routenplanung relevante Informationen, z.B. die Höhe und das Gewicht des Fahrzeugs},
  plural={Profile}
}

\newglossaryentry{route}{
  name={Route},
  description={Ein Weg zwischen zwei Punkten auf je einem Wegstück}
}
\newglossaryentry{schnellroute}{
  name={schnellste Route},
  description={die \gls{route} mit der geschätzt kürzesten Fahrzeit}
}
\newglossaryentry{wegbeschreibung}{
  name={Wegbeschreibung},
  description={Beschreibung einer \gls{route} durch eine Liste von Abbiegeanweisungen}
}
\newglossaryentry{arc}{
 name={Arc-Flags},
 description={eine Technik, um Routenberechnung zu beschleunigen}
}
\newglossaryentry{beschleunigung}{
 name={Beschleunigung},
 description={Ermöglicht eine schnellere Berechnung der angefragten Route},
 see={arc}
}

\newglossaryentry{kartendaten}{
  name={Kartendaten},
  description={Eine Datei im \gls{osm}-Format, die eine Karte in Form von Knoten, Kanten und Relationen mit weiteren Informationen enthält}
}
\newglossaryentry{standardprofil}{
  name={Standardprofil},
  description={Ein vorinstalliertes Profil für gängige \glspl{pkw} oder \glspl{lkw}},
  see={profil}  
}
\newglossaryentry{vorberechnung}{
  name={Vorberechnung},
  description={Wandelt die Kartendaten in ein effizienteres Format um und fügt auf dem Profil basierende Informationen hinzu, die eine schnelle Routenberechnung ermöglichen. Muss für jede neue \gls{kartendaten}/\gls{profil}-Kombination einmal ausgeführt werden}
}

\section{Einleitung}

Das Produkt ist an Fahrer von Kraftfahrzeugen gerichtet; es soll ihnen die Planung einer Reise erleichtern, indem es ihnen die \gls{schnellroute} zwischen zwei angegebenen Punkten berechnet und anzeigt.

Dabei wird besonders darauf Wert gelegt, dass die berechnete Route mit dem benutzten Fahrzeug auch tatsächlich gefahren werden kann. Dazu werden verschiedene Beschränkungen (Abbiegeeinschränkungen, Tonnage- und Höhenbeschränkungen, Verbot für bestimmte Fahrzeugtypen) bei der Routenberechnung berücksichtigt. Für die Berechnung der Route soll eine Beschleunigungstechnik eingesetzt werden.

Das Produkt soll auf jedem handelsüblichen Desktop-Computer, wie ihn die Mehrheit der Zielgruppe besitzt, lauffähig sein. Die Kartendaten stammen aus dem \gls{osm}-Projekt.
%Lucas

% Betreuer: Abbiegungbeschränkung, "Route berechnen", (gemacht)
\section{Zielbestimmung}
%Lucas

\subsection{Musskriterien}
% Betreuer: Ausführlicher.
\begin{description}
\oitem{MK}{karteAnzeigen} Die Software soll die Karte auf dem Bildschirm anzeigen.
\oitem{MK}{karteRendern} Kartenkacheln rendern (Nur Straßen ohne Namen) % nach fkt. Anforderungen verschieben?
\oitem{MK}{karteVerschieben} Der Benutzer kann die Kartenansicht verschieben.
\oitem{MK}{karteZoomen} Der Benutzer kann die Kartenansicht zoomen.
\oitem{MK}{startZielWaehlen} Der Benutzer kann Start- und Zielpunkt einer Route durch Mausklick auf der Karte auswählen.
\oitem{MK}{startZielZeigen} Der ausgewählte Start- und Zielpunkt wird auf der Karte angezeigt.
\oitem{MK}{routeBerechnen} Die Software soll die \gls{schnellroute} zwischen dem ausgewählten Start- und Zielpunkt berechnen.
\oitem{MK}{routeAnzeigen}  Die berechnete \gls{route} zwischen Start- und Zielpunkt wird auf der Karte angezeigt.
\end{description}

\subsection{Wunschkriterien}
\begin{description}
\oitem{WK}{beschlArc} Die Routenberechnung soll durch \gls{arc} beschleunigt werden.
\oitem{WK}{abbiegebeschraeung} Bei der Routenberechnung werden Abbiegeverbote aufgrund von Abbiegeeinschränkungen beachtet.
\oitem{WK}{einbahnstrasse} Bei der Routenberechnung werden Einbahnstraßen beachtet.
\oitem{WK}{fahrzeugtyp} Bei der Routenberechnung werden Durchfahrtsverbote für bestimmte Fahrzeugtypen beachtet.
\oitem{WK}{maximalgewicht} Bei der Routenberechnung werden Durchfahrtsverbote aufgrund des zulässigen Maximalgewichts und der Maximalhöhe des Fahrzeugs beachtet.
%\oitem{WK}{strassentyp}                                                                                                                   % TODO Welchen Straßentyp
\oitem{WK}{beschreibung} Für die berechnete \gls{route} soll eine \gls{wegbeschreibung} generiert werden.
\oitem{WK}{history} Der Benutzer Berechnungsanfragen in einer Favoritenliste speichern.
\oitem{WK}{export} Die berechnete \gls{route} kann als GPX-Datei exportiert werden.
\oitem{WK}{fahrzeit} Abschätzung der Fahrzeit einer \gls{route} unter Berücksichtung vom Benutzer festgelegter Durchschnittsgeschwindigkeiten und der zulässigen Höchstgeschwindigkeiten der befahrenen Straßenabschnitte sowie von Wartezeiten an Ampeln.
\oitem{WK}{osmKacheln} Der Benutzer kann sich optional \gls{osm}-Kacheln anstelle der eigenen gerenderten Kacheln anzeigen lassen.
\oitem{WK}{profilerstellen} Der Benutzer kann ein \gls{profil} erstellen
\oitem{WK}{profilloeschen} Der Benutzer kann ein \gls{profil} löschen
\oitem{WK}{profilaendern} Der Benutzer kann ein \gls{profil} ändern
\oitem{WK}{profile} Zwischen mehreren \glslink{profil}{Profilen} für die Vorberechnung wählen
\oitem{WK}{standardprofil} Es gibt ein \gls{standardprofil} für \glspl{pkw} und eins für \glspl{lkw} 
\oitem{WK}{koordinaten} Geokoordinaten des Start- und Zielpunkts können als Dezimalzahlen eingegeben werden.
\oitem{WK}{karteRendernStrassen} Straßennamen und Bezeichnungen in \ref{MK:karteRendern} einbeziehen.
\oitem{WK}{karteRendernOrtsnamen} Ortsnamen in \ref{MK:karteRendern} einbeziehen
\oitem{WK}{karteRendernDeko} Einbahnstraßen in \ref{MK:karteRendern} visualisieren
\oitem{WK}{karteRendernStrecken} Die Kartenkacheln werden gestreckt wenn sich der Zoom zwischen 2 Stufen befindet
% Betreuer: Profile(gemacht)

\end{description}

\subsection{Abgrenzungskriterien}
\begin{description}
\oitem{AK}{textSuche} Eine textuelle Suche nach Adressen oder \glspl{poi} ist nicht möglich.
\oitem{AK}{zwischenziele} Die Angabe von Zwischenzielen ist nicht möglich.
\oitem{AK}{standort} Eine Erkennung des aktuellen Standorts des Benutzers findet nicht statt.
\oitem{AK}{poi} Die Positionen von \glspl{poi} sowie Gebäude werden nicht auf der Karte angezeigt.
\oitem{AK}{verkehrszeichen} Ampeln und Verkehrszeichen werden nicht auf der Karte visualisiert.
\oitem{AK}{baustelle} Baustellen werden (über die in den Kartendaten enthaltenen Informationen hinaus) nicht berücksichtigt.
\oitem{AK}{stau} TMC-Verkehrsmeldungen zu Staus werden nicht berücksichtigt.
\oitem{AK}{fahrradFussBahn} Die Software ist kein Routenplaner für Fahrradfahrer, Fußgänger oder öffentliche Verkehrsmittel.
\oitem{AK}{altRoute} Alternativrouten können nicht berechnet werden.
\oitem{AK}{app} Die Software ist keine mobile Anwendung (App).
\oitem{AK}{webapp} Die Software ist keine Webanwendung.
\end{description}

\section{Produkteinsatz}
%Kevin

Das Produkt soll Autofahrern bei der Planung einer optimalen Fahrstrecke helfen.

\subsection{Anwendungsbereiche}
\begin{itemize}
\item Routenplanung
\end{itemize}

\subsection{Zielgruppe}
\begin{itemize}
\item Autofahrer
\item Beifahrer
\item Personen, welche Routen für diese zuteilen bzw. erstellen
\end{itemize}

\subsection{Betriebsbedingungen}
\begin{itemize}
\item Zu Hause oder in Büroumgebungen
\end{itemize}

\section{Produktumgebung}
%Kevin

\subsection{Software}\label{subsec:Software}

\begin{itemize}
\item Betriebssystem: Linux, Windows, andere Betriebssysteme mit Java $\geq$ 7
\item Java-Laufzeitumgebung: Version 7 oder neuer
\end{itemize}

\subsection{Hardware}

\begin{itemize}
\item Ein Standard-PC mit angeschlossener Maus und Tastatur
\item Dieser sollte über einen Farbbildschirm mit der Auflösung von mindestens 1920$\times$1080 Pixeln verfügen
\item Es muss genügend Arbeitsspeicher (mindestens 4 Gigabyte) und Festplattenkapazität (mindestens 20 Gigabyte freier Speicher) vorhanden sein
\item Es muss die \hyperref[subsec:Software]{oben} genannte Software auf dem Computer lauffähig sein und bereits installiert und konfiguriert sein
\end{itemize}

\section{Funktionale Anforderungen}
% Betreuer: Weniger Text(gemacht)
% Felix
\subsection{Kernfunktionen}
\begin{description}
\oitem{F}{berechneNächstenPunktAufKante}
Berechnen des nächsten Punkts auf einer Kante zu Geokoordinaten auf der Karte.
\oitem{F}{berechneRoute}
Berechnen einer Route zwischen zwei Punkten auf je einer Kante des Straßengraphs.
\end{description}
\subsection{Benutzerschnittstelle}
\begin{description}
\oitem{F}{berechneKachel}
Berechnen einer Bildkachel aus lokalen Kartendaten zu gegebener Detailstufe und gegebenem Ausschnitt.
\oitem{F}{bestimmeKnotenInAusschnitt}
Bestimmen aller Knoten aus einem gegebenen Kartenausschnitt. (Benötigt für \ref{F:berechneKachel})
\oitem{F}{fetcheOnlineKachel}
Aufbauen einer HTTP-Verbindung und anfordern von Online-Kacheln.
\oitem{F}{zeichneKachel}
Darstellen von Kartenkacheln der entsprechenden Zoomstufe an der richtigen Stelle im \gls{gui}.
\oitem{F}{bestimmeKoordinatenAusPunkt}
Bestimmen der Geokoordinaten zu einem gegebenem Pixel auf der angezeigten Karte.
\oitem{F}{eingabeZuAusschnitt}
Umrechnung von Benutzereingaben auf der Karte in neuen Kartenausschnitt.
\oitem{F}{zeigeRoute}
Anzeigen einer berechneten Route (\ref{F:berechneRoute}) auf den dargestellten Kacheln (\ref{F:zeichneKachel}). Beispiel in
Abbildung \ref{fig:mockupscreenshotmain}
\end{description}
\subsection{Wegbeschreibung}
\begin{description}
\oitem{F}{bestimmeAbbiegeAusRoute}
Bestimmen aller Abbiegevorgänge einer Route.
\oitem{F}{klassifiziereAbbiege}
Klassifiziere die Abbiegevorgänge (rechts, links, scharf rechts, scharf links, geradeaus, halblinks, halbrechts, halblinks, rechts halten, links halten) und Abbiegevorgänge im Kreisverkehr (Ausfahrten zählen).
\oitem{F}{bestimmeAnweisungAusAbbiege}
Bestimmen der Anweisung aus einem Abbiegevorgang.
\oitem{F}{beschreibungZuHtml}
Erzeugen einer HTML-Datei aus einer Wegbeschreibung.
\oitem{F}{routeZuNavi}
Erzeugen einer Datei mit Geokoordinaten für die Benutzung in Navigationsgeräten.
\end{description}
\subsection{Profilverwaltung}
\begin{description}
\oitem{F}{neuesProfil} Neues Profil anlegen
\oitem{F}{loescheProfil} Vorhandenes Profil löschen
\oitem{F}{aendereProfil} Vorhandenes Profil ändern
\end{description}

\section{Produktdaten}
% Betreuer: Darstellung und Berechnung nur aus eigenen Daten. Steht noch nicht da.
%Anastasia
\subsection{Profildaten}
In einem \gls{profil} werden gespeichert:
\begin{description}
\oitem{D}{fahrzeugTyp}
Fahrzeugtyp (\gls{pkw}, \gls{lkw}, Bus, Motorrad),
\oitem{D}{hoehe}
Höhe des Fahrzeugs,
\oitem{D}{gewicht}
Gewicht des Fahrzeugs,
\oitem{D}{geschwindigkeitAutobahn}
Durchschnittsgeschwindigkeit auf Autobahn/Schnellstraße,
\oitem{D}{geschwindigkeitLandstrasse}
Durchschnittsgeschwindigkeit auf Landstraße.
\end{description}

\subsection{Kartendaten}
Es werden gespeichert:
\begin{description}
\oitem{D}{osmDatei}
\gls{osm}-Dateien,
\oitem{D}{vorberechnung}
Datei mit den Ergebnissen der \gls{vorberechnung}.
\end{description}

\subsection{Historydaten}
Für die History wird gespeichert:
\begin{description}
\oitem{D}{history}
Start- und Zielpunkt aus den vorherigen Anfragen des Benutzers.  
\end{description}
\section{Systemmodell} % Betreuer: Paketdiagramm
\includegraphics[width=\linewidth]{Paketdiagramm}
%Anastasia
\section{Nichtfunktionale Anforderungen und Qualitätsziele}

\begin{description}
\oitem{NF}{routeFuenfSek} \ref{F:berechneRoute} (Berechnung der Route) ist schneller als fünf Sekunden.
\oitem{NF}{knotenHalbeSek} \ref{F:berechneNächstenPunktAufKante} benötigt höchstens 500 ms.
\oitem{NF}{knotenWaysAnzahl} Die Software muss Daten mit mindestens x Knoten und y Ways verarbeiten können.
\end{description}
% Dominic
\section{Benutzeroberfläche}
% Betreuer: weniger Klicks? Startknopf.
% Betreuer: Elemente durchnummerieren. Einzeln erklären.
\begin{figure}[h]
\centering
\includegraphics[width=0.7\linewidth]{mockup_screenshot_main}
\caption{Erster Entwurf}
\label{fig:mockupscreenshotmain}
\end{figure}
\begin{enumerate}
\item \gls{profil}-Button
\item Karten-Button
\item ausgewählte Karte, ausgewähltes \gls{profil}
\item ausgewählter Startpunkt, ausgewähltes Ziel (nur Text, keine Eingabe)
\item \gls{wegbeschreibung} mit Scrollbalken
\item Karte
\item Zoom-Buttons
\item Rechtsklick-Kontextmenü
\item Startpunkt
\item Ziel
\item \gls{route}
\end{enumerate}
\begin{figure}[h]
\centering
\includegraphics[width=0.7\linewidth]{mockup_screenshot_nicht_berechnet}
\caption{Keine Vorberechnung}
\label{fig:mockupscreenshotkeinevorberechnung}
\end{figure}
\begin{enumerate}
\item Warnmeldung
\item ausgegraute Karte
\item \gls{vorberechnung}-Button
\end{enumerate}
% Fabian

\section{Globale Testfälle und Szenarien}
% Betreuer: Vorbedinung, Aktionen, Nachbedingung
\subsection{Funktionssequenzen}
Bei allen Testfällen gilt als Vorbedingung, dass das Programm gestartet ist.
\begin{description}
\oitem{TF}{waehleErstenPunkt}\testfall
    {Karte ist sichtbar, Profil ist vorberechnet (für die aktuelle Karte)}
    {Auswählen des Start- bzw. Zielpunkts}
    {Start- bzw. Zielpunkt ist ausgewählt und wird auf der Karte angezeigt}
\oitem{TF}{waehleZweitenPunkt}\testfall
    {Karte ist sichtbar, Profil ist vorberechnet(für die aktuelle Karte),  Start- oder Zielpunkt ist ausgewählt}
    {Auswählen des anderen Punkts (Ziel- bzw. Startpunkts)}
    {Route für die Angaben ist berechnet und wird angezeigt}
\oitem{TF}{waehlePunktErneut}\testfall
    {Karte ist sichtbar, Profil ist vorberechnet(für die aktuelle Karte),  Start- und Zielpunkt sind ausgewählt}
    {Erneutes Auswählen des  Ziel- oder Startpunkts}
    {Route für die neuen Angaben ist berechnet und wird angezeigt}
\oitem{TF}{zieheKarte}\testfall
    {Karte ist sichtbar}
    {Ziehen der Karte}
    {Die Kartenposititon hat sich entsprechend verändert}
\oitem{TF}{zoomeRein}\testfall
    {Karte ist sichtbar}
    {Reinzoomen}
    {Ein Auschnitt der Karte wird größer und genauer dargestellt}
\oitem{TF}{zoomeRaus}\testfall
    {Karte ist sichtbar}
    {Rauszoomen}
    {Ein größerer Bereich der Karte wird angezeigt}
\oitem{TF}{verbieteWaehlePunkt}\testfall
    {Karte ist sichtbar, aktuelles Profil ist nicht vorberechnet (für die aktuelle Karte)}
    {Versuchen Start- bzw. Zielpunkte auszuwählen}
    {Kein Start- oder Zielpunkt ist ausgewählt}
\oitem{TF}{starteVorberechnung}\testfall
    {Karte ist sichtbar, aktuelles Profil ist nicht vorberechnet (für die aktuelle Karte)}
    {Vorberechnung auswählen, lange Warten}
    {Aktuelles Profil ist vorberechnet (für die aktuelle Karte)}
\oitem{TF}{profilAnlegen}\testfall
    {Das erforderliche Profil ist noch nicht vorhanden}
    {Neues Profil anlegen}
    {Das neue Profil ist ausgewählt, das neue Profil ist nicht vorberechnet(für die aktuelle Karte)}
\oitem{TF}{waehleProfil}\testfall
    {Das erforderliche Profil ist vorhanden, aber nicht ausgewählt}
    {Ein Profil wird ausgewählt}
    {Das Profil ist ausgewählt}
\oitem{TF}{loescheProfil}\testfall
    {Ein unbrauchbares Profil ist vorhanden}
    {Ein Profil wird gelöscht}
    {Das Profil ist nicht mehr vorhanden}
\oitem{TF}{importKarte}\testfall
    {Die erforderliche Karte ist veraltet oder exestiert nicht}
    {Eine neue Karte wird importiert}
    {Die neue Karte ist aktuell, kein Profil ist für die aktuelle Karte vorberechnet}
\oitem{TF}{waehleKarte}\testfall
    {Eine Karte ist ausgewählt}
    {Eine andere Karte wird ausgewählt}
    {Die andere Karte ist ausgewählt}
\oitem{TF}{speichereRoute}\testfall
    {Eine Route ist berechnet}
    {Die Route wird gespeichert}
    {Die Route ist abgespeichert}
\oitem{TF}{ladeRoute}\testfall
    {Eine Route ist abgespeichert}
    {Die abgespeicherte Route wird geladen}
    {Die abgespeicherte Route wird wieder angezeigt}
\oitem{TF}{standardProfil}\testfall
    {Ein \gls{standardprofil} exestiert}
    {Versuchen das \gls{standardprofil} zu löschen}
    {Das \gls{standardprofil} ist nicht gelöscht}
\oitem{TF}{einbahnstraße}\testfall
    {Karte ist sichtbar, Profil ist vorberechnet(für die aktuelle Karte), Karte enthält eine Einbahnstraße}
    {Als Startpunkt wird das Ende und als Ziel der Anfang der Einbahnstraße ausgewählt}
    {Ein Weg, der nicht diese Einbahnstraße beinhaltet, wird gefunden}

\end{description}
\subsection{Datenkonsistenzen}

\begin{description}
\oitem{TD}{datenProfilVorberechnung}
Eine Route kann nur berechnet werden, wenn \gls{kartendaten} vorliegen, ein \gls{profil} ausgewählt wurde und die \gls{vorberechnung} für dieses \gls{profil} für diese Karte abgeschlossen ist.
\oitem{TD}{gültigOsm}
\gls{kartendaten} müssen als gültige \gls{osm}-Datei vorliegen.
\oitem{TD}{karteProfilGelöscht}
Eine gespeicherte Route kann nicht angezeigt werden, wenn die zugehörigen Kartendaten und das verwendete \gls{profil} nicht mehr vorhanden sind.
\oitem{TD}{löscheNichtStandard}
Das \gls{standardprofil} kann nicht gelöscht werden.
\oitem{TD}{profil}
Es ist ein \gls{profil} ausgewählt.
\oitem{TD}{karte}
Es ist eine Karte ausgewählt.
\end{description}
% Fabian
\section{Entwicklungsumgebung}
\begin{description}
\item[Teamkommunikation] E-Mail-Verteiler, Skype
\item[Dokumentation] \LaTeX{}
\item[UML-Planungswerkzeug] UMLet
\item[IDE] Eclipse
\item[Qualitätssicherung] JUnit, CodeCover
\item[Versionskontrolle] Git
\end{description}
% Kevin

\glsaddallunused
\printglossary[type=main, title={Glossar}, toctitle={Glossar}]

\end{document}
