% das Papierformat zuerst
\documentclass[a4paper, 11pt]{article}
\usepackage[utf8]{inputenc}
\usepackage[T1]{fontenc}
\usepackage[ngerman]{babel}

 
\title{Pflichtenheft}
%\author{}
%\date{}

% hier beginnt das Dokument
\begin{document}
\shorthandoff{"}

\maketitle
\newpage
\tableofcontents
\newpage


\section{Einleitung}

\section{Zielbestimmung}

\section{Produkteinsatz}

\section{Produktumgebung}

\section{Funktionale Anforderungen}

\subsection{Musskriterien}
\begin{itemize}
\item "Ausreichend komfortables GUI"
\begin{itemize}
\item Rechtsklick auf die Karte, "hier Start", "hier Ziel"
\item Karte anzeigen
\item Karte ziehen/zoomen
\item Weg anfragen
\item Anzeigen der Route
\end{itemize}
\item Berechnen einer Route zwischen 2 Punkten auf der Karte.
\end{itemize}
\subsection{Wunschkriterien}
\begin{itemize}
\item Umsetzen möglichst vieler Beschränkungen (Staßentyp, Abbiegebeschränkung, 
Maximalgewicht, Einbahnstraße)

\item Wegbeschreibung anzeigen. (Kreisverkehre, Zählen)
\item Suchanfragen speichern. History/Favoriten

\item Suchen von Adressen (+ POIs)
\item Zwischenziele, (mit DnD oder manuell)
\item Such Type-ahead
\item Offline-Lite

\item Tankstellen/Hotels etc. POIs an der Strecke.
\end{itemize}
\subsection{Abgerenzungskriterien}

\begin{itemize}
\item Kein Routenplaner für Flugzeuge, Einräder, Taucher oder etc.
\item Keine Moblie App.
\item Keine Webanwendung
\item Keine Alternativroute sofort anzeigen

\end{itemize}


\section{Produktdaten}

\section{Systemmodell}

\section{Nichtfunktionale Anforderungen}

\begin{enumerate}
\item Such der Route schneller als 5 Sekunden.

\end{enumerate}

\section{Benutzeroberfläche}

\section{Qualitätsziele}

\section{Globale Testfälle}

\section{Entwicklungsumgebung}

\section{Glossar}

\end{document}
