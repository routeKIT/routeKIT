% das Papierformat zuerst
\documentclass[a4paper, 11pt]{article}
\usepackage[utf8]{inputenc}
\usepackage[T1]{fontenc}
\usepackage[ngerman]{babel}
\usepackage{hyperref} % clickable refs
\usepackage[toc, numberedsection, style=altlist]{glossaries}
\makeglossary

 
\title{Pflichtenheft}
%\author{}
%\date{}

%Hack for referencing labels
\makeatletter
\let\orgdescriptionlabel\descriptionlabel
\renewcommand*{\descriptionlabel}[1]{%
  \let\orglabel\label
  \let\label\@gobble
  \edef\@currentlabel{#1}%
  \let\label\orglabel
  \orgdescriptionlabel{#1}%
}
\makeatother
% End: Hack for referencing labels

% Glossar: alle Einträge, aber ohne extra Referenzen
% http://tex.stackexchange.com/questions/115635/glossaries-suppress-pages-when-using-glsaddall
\newcommand*{\glsgobblenumber}[1]{}
\makeatletter
\newcommand*{\glsaddnp}[2][]{
  \glsdoifexists{#2}{
    \def\@glsnumberformat{glsgobblenumber}
    \edef\@gls@counter{\csname glo@#2@counter\endcsname}
    \setkeys{glossadd}{#1}
    \@gls@saveentrycounter
    \@do@wrglossary{#2}
  }
}
\newcommand{\glsaddallunused}[1][]{
  \edef\@glo@type{\@glo@types}
  \setkeys{glossadd}{#1}
  \forallglsentries[\@glo@type]{\@glo@entry}{
    \ifglsused{\@glo@entry}{}{
      \glsaddnp[#1]{\@glo@entry}}}
}
\makeaddother

\renewcommand{\glsnamefont}[1]{\mdseries #1} % glossary entries shouldn’t be bold

% Glossar

% So sieht ein Glossar-Eintrag aus:
%
%\newglossaryentry{dijkstra}{
%  name={Dijkstra’s Algorithmus},
%  description={ein Algorithmus, um den optimalen Pfad in einem gerichteten Graphen zu finden}
%}
%\newglossaryentry{arc}{
%  name={Arc-Flags},
%  description={eine Technik, um Routenberechnung zu beschleunigen},
%  see=[siehe auch]{dijkstra}
%}
%
% Und so kann er im Dokument verwendet werden:
%
% lorem ipsum dolor sit \gls{arc}, consectetur
%
% End: Glossar

% hier beginnt das Dokument
\begin{document}
\shorthandoff{"}

\maketitle
\newpage
\tableofcontents
\newpage


\section{Einleitung}
% FIXME less buzzwords, more content
In unserer zunehmend vernetzten Welt ist persönliche Mobilität von hoher Bedeutung für berufliche Flexibilität und persönliche Selbstverwirklichung. Dabei ist Unterstützung durch IT-Systeme entscheidend, um den Menschen das Leben so einfach wie möglich zu machen. Ein Baustein dieser Unterstützung sind Werkzeuge und Programme, um eine unter bestimmten Kriterien optimale Route zwischen zwei Punkten zu finden.
%Lucas

\section{Zielbestimmung}
Das Produkt ist an Autofahrer und/oder deren Mitfahrer gerichtet; es soll ihnen die Planung einer Reise erleichtern, indem es ihnen die optimale (schnellste) Route zwischen zwei Punkten berechnet und anzeigt.
Dabei wird besonders darauf Wert gelegt, dass die gewählte Route auch gefahren werden kann; dazu werden verschiedene Beschränkungen (Abbiegebeschränkungen, Beschränkungen aufgrund der Größe oder des Gewichts des Fahrzeugs, ...) bei der Berechnung der Route berücksichtigt.
%Lucas

\subsection{Musskriterien}
\begin{itemize}
\item "Ausreichend komfortables GUI"
\begin{itemize}
\item Rechtsklick auf die Karte, "hier Start", "hier Ziel"
\item Karte anzeigen
\item Karte ziehen/zoomen
\item Weg anfragen
\item Anzeigen der Route
\end{itemize}
\item Berechnen einer Route zwischen 2 Punkten auf der Karte.
\end{itemize}
\subsection{Wunschkriterien}
\begin{itemize}
\item Umsetzen möglichst vieler Beschränkungen (Staßentyp, Abbiegebeschränkung, 
Maximalgewicht, Einbahnstraße) trotz Beschleunigung. Profile für die Vorberechnung.

\item Wegbeschreibung anzeigen. (Kreisverkehre, Zählen)
\item Berechnungsanfragen speichern. History/Favoriten
\item Export, Drucken von Route.

\item Suchen von Adressen (+ POIs)

\item Zwischenziele, (mit DnD oder manuell)
\item Optional OSM-Kacheln
\item Tankstellen/Hotels etc. POIs an der Strecke.
\end{itemize}
\subsection{Abgerenzungskriterien}

\begin{itemize}
\item Kein Routenplaner für Fahrrad, Fußgänger, Bahnfahrer oder etc.
\item Keine Mobile App.
\item Keine Webanwendung
\item Keine Alternativroute sofort anzeigen

\end{itemize}

\section{Produkteinsatz}
%Kevin
\section{Produktumgebung}
%Kevin
\section{Funktionale Anforderungen}
% Felix
\subsection{Kernfunktionen}
\begin{description}
\item [/F10/]
Berechnen des nächsten Knotens zu Geokoordinaten auf der Karte.\\
Für das Auflösen eines geklickten Punkts in einen Knoten, wird Programmcode benötigt, der einen Punkt in Geokoordinaten auf den nächstliegenden Weg projeziert.
\item[/F20/]
Berechnen einer Route zwischen zwei Knoten des Straßengraphs.\\
Für die Nutzung der Programms wird ein optimierter Routenplanungsalogrithmus benötigt, der die Kernfunktionalität darstellt.
\end{description}
\subsection{Benutzerschnittstelle}
\begin{description}
\item[/F30/]
Berechnen einer Bildkachel aus lokalen Kartendaten.\\
Darstellung der lokalen Kartendaten in verschiedenen Zoom- und Genauigkeitsstufen. Dazu wird zunächst \ref{F40} benötigt.
\item[/F40/\label{F40}]
Bestimmen aller Knoten aus einem gegebenen Kartenausschnitt.\\

\item[/F50/]
Aufbauen einer HTTP-Verbindung und anfordern von Online-Kacheln
\item[/F60/]
Darstellen von Kartenkacheln
\item[/F70/]
Bestimmen der Geokoordinaten zu einer gegebenen Positon auf der angezeigten Karte.
\item[/F80/]
Umrechnung von Benutzereingaben auf der Karte in neuen Kartenausschnitt.
\item[/F90/]
Anzeigen einer berechneter Route auf den dargestellten Kacheln.
\end{description}
\subsection{Wegbeschreibung}
\begin{description}
\item[/F100/]
Bestimmen aller Abbiegevorgänge einer Route
\item[/F110/]
Bestimmen der Anweisung aus einem Abbiegevorgang
\item[/F120/]
Erzeugen einer HTML-Datei aus einer Wegbeschreibung
\item[/F130/]
Erzeugen einer Datei mit Geokoordinaten für die Benutzung in Navis.
\end{description}
\subsection{Suche}
\begin{description}
\item[/F140/]
Erstellen eines Suchindexes für Addressen.

\end{description}
\section{Produktdaten}
%Anastasia
\section{Systemmodell}
%Anastasia
\section{Nichtfunktionale Anforderungen}

\begin{enumerate}
\item Suchen der Route schneller als 5 Sekunden. (  mit Arcflags)

\end{enumerate}
% Dominic
\section{Benutzeroberfläche}
% Fabian
\section{Qualitätsziele}
% Dominic
\section{Globale Testfälle und Szenarien}
% Fabian
\section{Entwicklungsumgebung}
% Kevin

\glsaddallunused
\printglossary[type=main, title={Glossar}, toctitle={Glossar}]

\end{document}
