% das Papierformat zuerst
\documentclass[a4paper, 11pt]{article}
\usepackage[utf8]{inputenc}
\usepackage[T1]{fontenc}
\usepackage[ngerman]{babel}
\usepackage{hyperref} % clickable refs
\usepackage{graphicx}

 
\title{Pflichtenheft}
%\author{}
%\date{}

%Hack for referencing labels
\makeatletter
\let\orgdescriptionlabel\descriptionlabel
\renewcommand*{\descriptionlabel}[1]{%
  \let\orglabel\label
  \let\label\@gobble
  \edef\@currentlabel{#1}%
  \let\label\orglabel
  \orgdescriptionlabel{#1}%
}
\makeatother
% End: Hack for referencing labels

% hier beginnt das Dokument
\begin{document}
\shorthandoff{"}

\maketitle
\newpage
\tableofcontents
\newpage


\section{Einleitung}
% FIXME less buzzwords, more content
In unserer zunehmend vernetzten Welt ist persönliche Mobilität von hoher Bedeutung für berufliche Flexibilität und persönliche Selbstverwirklichung. Dabei ist Unterstützung durch IT-Systeme entscheidend, um den Menschen das Leben so einfach wie möglich zu machen. Ein Baustein dieser Unterstützung sind Werkzeuge und Programme, um eine unter bestimmten Kriterien optimale Route zwischen zwei Punkten zu finden.
%Lucas

\section{Zielbestimmung}
Das Produkt ist an Autofahrer und/oder deren Mitfahrer gerichtet; es soll ihnen die Planung einer Reise erleichtern, indem es ihnen die optimale (schnellste) Route zwischen zwei Punkten berechnet und anzeigt.
Dabei wird besonders darauf Wert gelegt, dass die gewählte Route auch gefahren werden kann. Dazu werden verschiedene Beschränkungen (Abbiegebeschränkungen, Beschränkungen aufgrund der Größe oder des Gewichts des Fahrzeugs, ...) bei der Berechnung der Route berücksichtigt.

Das Produkt soll auf jedem handelsüblichen Desktop-Computer, wie ihn die Mehrheit der Zielgruppe besitzt, lauffähig sein. Um Lauffähigkeit unter verschiedenen Betriebssystemen zu erreichen, wird die Java-Plattform eingesetzt.

Die Kartendaten stammen aus dem OpenStreetMap-Projekt. Aufgrund der schieren Menge an Daten ist eine weltweite Routenplanung nicht praktikabel, und der Bedarf dafür dürfte sich auch in Grenzen halten. Als Maßstab für Kartengrößen, mit denen das Programm noch umgehen können soll, wählen wir daher die Deutschlandkarte.
%Lucas

\subsection{Musskriterien}
\begin{itemize}
\item "Ausreichend komfortables GUI"
\begin{itemize}
\item Rechtsklick auf die Karte, "hier Start", "hier Ziel"
\item Karte anzeigen
\item Karte ziehen/zoomen
\item Weg anfragen
\item Anzeigen der Route
\end{itemize}
\item Berechnen einer Route zwischen 2 Punkten auf der Karte.
\end{itemize}
\subsection{Wunschkriterien}
\begin{itemize}
\item Umsetzen möglichst vieler Beschränkungen (Staßentyp, Abbiegebeschränkung, 
Maximalgewicht, Einbahnstraße) trotz Beschleunigung. Profile für die Vorberechnung.

\item Wegbeschreibung anzeigen. (Kreisverkehre, Zählen)
\item Berechnungsanfragen speichern. History/Favoriten
\item Export, Drucken von Route.

\item Suchen von Adressen (+ POIs)

\item Zwischenziele, (mit DnD oder manuell)
\item Optional OSM-Kacheln
\item Tankstellen/Hotels etc. POIs an der Strecke.
\end{itemize}
\subsection{Abgerenzungskriterien}

\begin{itemize}
\item Kein Routenplaner für Fahrrad, Fußgänger, Bahnfahrer oder etc.
\item Keine Mobile App.
\item Keine Webanwendung
\item Keine Alternativroute sofort anzeigen

\end{itemize}

\section{Produkteinsatz}
%Kevin
\section{Produktumgebung}
%Kevin
\section{Funktionale Anforderungen}
% Felix
\subsection{Kernfunktionen}
\begin{description}
\item [/F10/]
Berechnen des nächsten Knotens zu Geokoordinaten auf der Karte.\\
Für das Auflösen eines geklickten Punkts in einen Knoten, wird Programmcode benötigt, der einen Punkt in Geokoordinaten auf den nächstliegenden Weg projeziert.
\item[/F20/]
Berechnen einer Route zwischen zwei Knoten des Straßengraphs.\\
Für die Nutzung der Programms wird ein optimierter Routenplanungsalogrithmus benötigt, der die Kernfunktionalität darstellt.
\end{description}
\subsection{Benutzerschnittstelle}
\begin{description}
\item[/F30/]
Berechnen einer Bildkachel aus lokalen Kartendaten.\\
Darstellung der lokalen Kartendaten in verschiedenen Zoom- und Genauigkeitsstufen. Dazu wird zunächst \ref{F40} benötigt.
\item[/F40/\label{F40}]
Bestimmen aller Knoten aus einem gegebenen Kartenausschnitt.\\
Mit einer angepassten Datenstruktur sollen solche Anfragen effizient abgearbeitet werden.
\item[/F50/]
Aufbauen einer HTTP-Verbindung und anfordern von Online-Kacheln.\\
Online-Kacheln sollen in Echtzeit von einem gegebenen HTTP-Server.
\item[/F60/\label{F60}]
Darstellen von Kartenkacheln\\
Die Kacheln der entsprechenden Zoomstufe sollen an der richtigen Stelle im GUI angezeigt werden.
\item[/F70/\label{F70}]
Bestimmen der Geokoordinaten zu einer gegebenen Position auf der angezeigten Karte.\\
Mithilfe der aktuellen Zoomstufe und der Kartenpostiton soll aus dem geklickten Pixel die entprechender Geokoordinaten berechnet werden.
\item[/F80/]
Umrechnung von Benutzereingaben auf der Karte in neuen Kartenausschnitt.\\
Benutzeraktionen wie Drag Events oder Mausrad Events sollen auf die aktuelle Kartenposition/Zoomstufe angewendet werden. (Benötigt evtl. \ref{F70})
\item[/F90/]
Anzeigen einer berechneter Route auf den dargestellten Kacheln.\\
Zusätzlich zu \ref{F60} soll auf der selben Fläche die berechnete Route visuell dargestellt werden (Beispiel in  Abbildung \ref{fig:mockupscreenshot}).
\end{description}
\subsection{Wegbeschreibung}
\begin{description}
\item[/F100/]
Bestimmen aller Abbiegevorgänge einer Route.\\
Die entsprechenden Abbiegevorgänge müssen dabei klassifiziert werden (Richtung, ist Kreisverkehr).
\item[/F110/]
Bestimmen der Anweisung aus einem Abbiegevorgang.\\
Aus definierten extern gespeicherten Textvorlagen soll die richtige Beschreibung generiert werden.
\item[/F120/]
Erzeugen einer HTML-Datei aus einer Wegbeschreibung\\
Die Wegbeschreibung soll nachher aus dem Programm in eine HTML-Datei exportiert werden.
\item[/F130/]
Erzeugen einer Datei mit Geokoordinaten für die Benutzung in Navis.
\end{description}
\subsection{Suche}
\begin{description}
\item[/F140/]
Erstellen eines Suchindexes für Addressen.

\end{description}
\section{Produktdaten}
%Anastasia
\subsection{Profildaten}
\begin{description}
\item [/D10/\label{D10}]
Es wird der Typ des Fahrzeugs (PKW, LKW, Bus) gespechert.
\item [/D20/]
Für ein Fahrzeug vom Typ  PKW wird die vom Fahrer angegebene Durchschnittsgeschwindigkeit gespeichert.
\item [/D30/]
Für ein Fahrzeug vom Typ  LKW werden folgende Daten gespeichert:\\
Länge, Breite, Höhe und Gewicht des Autos, maximale auf einer Autobahn erlaubte Geschwindigkeit.
\item [/D40/]
Für ein Fahrzeug vom Typ Bus werden folgende Daten gespeichert:\\
Höhe und maximale auf einer Autobahn erlaubte Geschwindigkeit.
\end{description}

\subsection{Kartendaten}
\begin{description}
\item [/D50/\label{D50}]
Es wird eine OSM-Datei gespeichert.
\item [/D60/\label{D60}]
Für eine bestimmte OSM-Datei wie in \ref{D50} und ein bestimmten Typ des Fahrzeugs wie in \ref{D10} wird ein Graph gespeichert.
\item [/D70/]
Für einen Graphen wie in \ref{D60} werden  folgende Daten gespeichert:\\
Knoten und Kanten des Graphen, Gewichte der Kanten, zusätzliche Information über die Angehörigkeit einer Kante zum kürzesten Pfad in Form von Arc-Flags.
\item [/D80/]
Für jeden Knoten werden geographischen Koordinaten (Länge und Breite) und eine unikale Identifikationsnummer.
\end{description}

\subsection{Historydaten}
\begin{description}
\item [/D90/]
Für eine bestimmte OSM-Datei wie in \ref{D50} und ein bestimmten Typ des Fahrzeugs wie in \ref{D10} wird eine bestimmte Anzahl von Ergebnissen der vorherigen Anfragen des Benutzers gespeichert, was als History bezeichnet wird.  
\end{description}
\section{Systemmodell}
\includegraphics[width=\linewidth]{Systemmodell1}
%Anastasia
\section{Nichtfunktionale Anforderungen}

\begin{enumerate}
\item Suchen der Route schneller als 5 Sekunden. (  mit Arcflags)

\end{enumerate}
% Dominic
\section{Benutzeroberfläche}
\begin{figure}
\centering
\includegraphics[width=0.7\linewidth]{mockup_screenshot}
\caption{Erster Entwurf}
\label{fig:mockupscreenshot}
\end{figure}
% Fabian
\section{Qualitätsziele}
% Dominic
\section{Globale Testfälle und Szenarien}
\subsection{Funktionssequenzen}
\begin{description}
\item[/TF10/]
Start auswählen, Ziel auswählen, Route berechnen
\item[/TF20/]
Karte bewegen, heranzoomen, Karte bewegen, herauszoomen
\item[/TF30/]
Profilerstellung, Vorberechnung, Route berechnen, Profil löschen
\item[/TF40/]
Route berechnen, Profil wechseln, Route berechnen
\item[/TF50/]
Neue Kartendaten laden, Vorberechnung, Route berechnen
\item[/TF60]
Route berechnen, Karte wechseln, Route berechnen
\item[/TF70/]
Route berechnen, Route speichern, andere Route berechnen, Kartenausschnitt verändern, gespeicherte Route laden
\end{description}
\subsection{Datenkonsistenzen}
\begin{description}
\item[/TD10/]
Eine Route kann nur berechnet werden, wenn Kartendaten vorliegen, ein Profil ausgewählt wurde und die Vorberechnung für dieses Profil für diese Karte abgeschlossen ist.
\item[/TD20/]
Kartendaten müssen als gültige osm-Datei vorliegen.
\item[/TD30/]
Eine gespeicherte Route kann nicht angezeigt werden, wenn die zugehörigen Kartendaten und das verwendete Profil nicht mehr vorhanden sind.
\item[/TD40/]
Das Standardprofil kann nicht gelöscht werden.
\end{description}
% Fabian
\section{Entwicklungsumgebung}
\begin{description}
  \item[UML-Planungswerkzeug] UMLet
  \item[IDE] Eclipse
  \item[Qualitätssicherung] JUnit, CodeCover
\end{description}
% Kevin
\section{Glossar}
\begin{description}
\item[Kartendaten] Eine Datei im osm-Format, die eine Karte in Form von Knoten, Kanten und Relationen mit weiteren Informationen enthält
\item[Profil]
Enthält für die Routenplanung relevante Informationen, z.B. die Höhe und das Gewicht des Fahrzeugs
\item[Route]
Ein Weg zwischen zwei Punkten, der auf der Karte angezeigt wird
\item[Standardprofil]
Ein vorinstalliertes Profil für gängige PKWs
\item[Vorberechnung] Wandelt die Kartendaten in ein effizienteres Format um und fügt auf dem Profil basierende Informationen hinzu, die eine schnelle Routenberechnung ermöglichen. Muss für jede neue Karte/Profil-Kombination einmal ausgeführt werden.
\end{description}
\end{document}
