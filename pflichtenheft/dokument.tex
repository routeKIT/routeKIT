% das Papierformat zuerst
\documentclass[a4paper, 11pt]{article}
\usepackage[margin=3cm]{geometry}
\usepackage[utf8]{inputenc}
\usepackage[T1]{fontenc}
\usepackage[ngerman]{babel}
\usepackage{hyperref} % clickable refs
\usepackage{graphicx}
\usepackage[toc, numberedsection, style=altlist]{glossaries}
\makeglossary

 
\title{Pflichtenheft}
\author{Kevin Birke \\ Felix Dörre \\ Fabian Hafner \\ Lucas Werkmeister \\ Dominic Ziegler \\ Anastasia Zinkina}
%\date{}

%Hack for referencing labels
\makeatletter
\def\namedlabel#1#2{\begingroup
    #2%
    \def\@currentlabel{#2}%
    \phantomsection\label{#1}\endgroup
}
\makeatother
% End: Hack for referencing labels

% Glossar: alle Einträge, aber ohne extra Referenzen
% http://tex.stackexchange.com/questions/115635/glossaries-suppress-pages-when-using-glsaddall
\newcommand*{\glsgobblenumber}[1]{}
\makeatletter
\newcommand*{\glsaddnp}[2][]{
  \glsdoifexists{#2}{
    \def\@glsnumberformat{glsgobblenumber}
    \edef\@gls@counter{\csname glo@#2@counter\endcsname}
    \setkeys{glossadd}{#1}
    \@gls@saveentrycounter
    \@do@wrglossary{#2}
  }
}
\newcommand{\glsaddallunused}[1][]{
  \edef\@glo@type{\@glo@types}
  \setkeys{glossadd}{#1}
  \forallglsentries[\@glo@type]{\@glo@entry}{
    \ifglsused{\@glo@entry}{}{
      \glsaddnp[#1]{\@glo@entry}}}
}
\makeatother

\renewcommand{\glsnamefont}[1]{\mdseries #1} % glossary entries shouldn’t be bold

% Glossar

% So sieht ein Glossar-Eintrag aus:
%
%\newglossaryentry{dijkstra}{
%  name={Dijkstra’s Algorithmus},
%  description={ein Algorithmus, um den optimalen Pfad in einem gerichteten Graphen zu finden}
%}
%\newglossaryentry{arc}{
%  name={Arc-Flags},
%  description={eine Technik, um Routenberechnung zu beschleunigen},
%  see={dijkstra}
%}
%
% Und so kann er im Dokument verwendet werden:
%
% lorem ipsum dolor sit \gls{arc}, consectetur
%
% End: Glossar

% usage: \counteditem{prefix}{refName} -> item '/prefixXX/' with label 'prefix:refName' (where XX is counted in increments of 10)
\makeatletter
\newcommand{\oitem}[2]{
  % define the counter
  \@ifundefined{c@oitem#1}{\newcounter{oitem#1}}{} % initialized at 0
  \addtocounter{oitem#1}{10}
  \item[\namedlabel{#1:#2}{/#1\arabic{oitem#1}/}]
}
\makeatother

\newcommand{\testfall}[3]{
  \begin{description}
    \item[Vorbedingungen] #1
    \item[Ablauf] #2
    \item[Erwartetes Ergebnis] #3
  \end{description}
}

% Betreuer: Rand reduzieren. (gemacht)
% Betreuer: Wichtig, viel Kriterien, GUI, Testfälle
% Betreuer: Präzise!!!
% hier beginnt das Dokument
\begin{document}
\shorthandoff{"}

% place a symbol before clickable links
% this has to come *after* \begin{document} because hyperref installs a \AtBeginDocument hook that updates the ref command.
\newcommand{\refsymbol}[0]{\scalebox{0.5}{$\nearrow$}}
\let\oldref\ref
\renewcommand{\ref}[1]{\refsymbol\oldref{#1}}
\let\oldgls\gls
\renewcommand{\gls}[1]{\refsymbol\oldgls{#1}}
\let\oldglslink\glslink
\renewcommand{\glslink}[2]{\refsymbol\oldglslink{#1}{#2}}

\maketitle
\newpage
\tableofcontents
\newpage
% Betreuer: Berechnung von Punkt auf Kante zu Punkt auf Kante.  (KdBaum über Kanten) (gemackt)
% Betreuer: Profile exakt welcher Typ

% alle Glossareintraege
\newglossaryentry{profil}{
  name={Profil},
  description={Enthält für die Routenplanung relevante Informationen, z.B. die Höhe und das Gewicht des Fahrzeugs},
  plural={Profile}
}

\newglossaryentry{route}{
  name={Route},
  description={Ein Weg zwischen zwei Punkten auf je einem Wegstück}
}
\newglossaryentry{schnellroute}{
  name={schnellste Route},
  description={die \gls{route} mit der geschätzt kürzesten Fahrzeit}
}

\newglossaryentry{arc}{
 name={Arc-Flags},
 description={eine Technik, um Routenberechnung zu beschleunigen}
}
\newglossaryentry{beschleunigung}{
 name={Beschleunigung},
 description={Ermöglicht eine schnellere Berechnung der angefragten Route},
 see={arc}
}

\newglossaryentry{kartendaten}{
  name={Kartendaten},
  description={Eine Datei im osm-Format, die eine Karte in Form von Knoten, Kanten und Relationen mit weiteren Informationen enthält}
}
\newglossaryentry{standardprofil}{
  name={Standardprofil},
  description={Ein vorinstalliertes Profil für gängige PKWs}
}
\newglossaryentry{vorberechnung}{
  name={Vorberechnung},
  description={Wandelt die Kartendaten in ein effizienteres Format um und fügt auf dem Profil basierende Informationen hinzu, die eine schnelle Routenberechnung ermöglichen. Muss für jede neue Karte/Profil-Kombination einmal ausgeführt werden}
}


\section{Einleitung}

Das Produkt ist an Autofahrer und/oder deren Mitfahrer gerichtet; es soll ihnen die Planung einer Reise erleichtern, indem es ihnen die \gls{schnellroute} zwischen zwei eingegebenen Punkten berechnet und anzeigt.
Dabei wird besonders darauf Wert gelegt, dass die gewählte Route auch gefahren werden kann. Dazu werden verschiedene Beschränkungen (Abbiegebeschränkungen, Beschränkungen aufgrund der Größe oder des Gewichts des Fahrzeugs,~...) bei der Berechnung der Route berücksichtigt.

Das Produkt soll auf jedem handelsüblichen Desktop-Computer, wie ihn die Mehrheit der Zielgruppe besitzt, lauffähig sein. Die Kartendaten stammen aus dem OpenStreetMap-Projekt.
%Lucas

% Betreuer: Abbiegungbeschränkung, "Route berechnen", (gemacht)
\section{Zielbestimmung}
%Lucas

\subsection{Musskriterien}
% Betreuer: Ausführlicher.
\begin{description}
\oitem{MK}{karteAnzeigen} Karte anzeigen
\oitem{MK}{karteRendern} Kartenkacheln rendern (Nur Straßen ohne Namen)
\oitem{MK}{karteVerschieben} Karte verschieben
\oitem{MK}{karteZoomen} Karte zoomen
\oitem{MK}{startZielWählen} Start- und Zielpunkt auswählen % Rechtsklick
\oitem{MK}{startZielZeigen} Start- und Zielpunkt auf der Karte anzeigen
\oitem{MK}{routeAnfragen}  \gls{schnellroute} zwischen Start- und Zielpunkt anfragen
\oitem{MK}{routeBerechnen}  \gls{schnellroute} zwischen Start- und Zielpunkt berechnen
\oitem{MK}{routeAnzeigen}  \gls{schnellroute} zwischen Start- und Zielpunkt auf der Karte anzeigen
\end{description}
\subsection{Wunschkriterien}
\begin{description}

\oitem{WK}{beschlArc} \gls{beschleunigung} durch \gls{arc} bei der Berechnung der Route
\oitem{WK}{fahrzeugtyp} Fahrverbot für einen Fahrzeugtyp durch eine Staße beachten
%\oitem{WK}{stassentyp}                                                                                                                   % TODO Welchen Straßentyp
\oitem{WK}{abbiegebeschraeung} Abbiegeverbot durch Abbiegebeschränkung beachten.
\oitem{WK}{maximalgewicht} Fahrverbot durch Maximalgewicht beachten
\oitem{WK}{einbahnstrasse} Fahrverbot durch Einbahnstraße beachten
\oitem{WK}{hoechstgeschwindigkeit} Höchstgeschwindigkeit beachten
\oitem{WK}{beschreibung} Wegbeschreibung anzeigen
\oitem{WK}{history} Berechnungsanfragen speichern. History/Favoriten
\oitem{WK}{export} Route exportieren
\oitem{WK}{fahrzeit} Schätzung der Fahrzeit einer \gls{route} unter Beachtung von Wartezeiten an Ampeln, maximalen Geschwindigkeiten, vom Benutzer festgelegten Durchschnittsgeschwindigkeiten
\oitem{WK}{osmKacheln} Optional OSM-Kacheln anstelle des eigenen Renderers verwenden
\oitem{WK}{profilerstellen} Der Benutzer kann ein \gls{profil} erstellen
\oitem{WK}{profilloeschen} Der Benutzer kann ein \gls{profil} löschen
\oitem{WK}{profilaendern} Der Benutzer kann ein \gls{profil} ändern
\oitem{WK}{profile} Zwischen mehreren \glslink{profil}{Profilen} für die Vorberechnung wählen
\oitem{WK}{koordinaten} Geokoordinaten des Start- und Zielpunkts können als Dezimalzahlen eingegeben werden.
\oitem{WK}{karteRendernStrassen} Straßennamen und Bezeichnungen in \ref{MK:karteRendern} einbeziehen.
\oitem{WK}{karteRendernOrtsnamen} Ortsnamen in \ref{MK:karteRendern} einbeziehen
\oitem{WK}{karteRendernDeko} Einbahnstraßen in \ref{MK:karteRendern} visualisieren
\oitem{WK}{karteRendernStrecken} Die Kartenkacheln werden gestreckt wenn sich der Zoom zwischen 2 Stufen befindet
% Betreuer: Profile(gemacht)

\end{description}
\subsection{Abgerenzungskriterien}

\begin{description}
\oitem{AK}{textSuche} Keine textuelle Suche
\oitem{AK}{zwischenziele} Keine Angabe von Zwischenzielen möglich
\oitem{AK}{standort} Keine Erkennung des aktuellen Standorts des Benutzers
\oitem{AK}{poi} Keine Informationen  über Positionen von POI`s, Hotels, Tankstellen, Geschäften etc.
\oitem{AK}{baustelle} Keine Berücksichtigung von Baustellen
\oitem{AK}{Stau} Keine Berücksichtigung von Staus
\oitem{AK}{verkehrszeichen} Keine Visualisierung von Ampeln und Verkehrszeichen auf der Karte
\oitem{AK}{fahrradFussBahn} Kein Routenplaner für Fahrrad, Fußgänger, öffentliche Verkehrsmittel etc.
\oitem{AK}{app} Keine mobile App
\oitem{AK}{webapp} Keine Webanwendung
\oitem{AK}{altRoute} Keine Alternativrouten berechnen

\end{description}

\section{Produkteinsatz}
%Kevin

Das Produkt soll Autofahrern bei der Planung einer optimalen Fahrstrecke helfen.

\subsection{Anwendungsbereiche}
\begin{itemize}
\item Routenplanung
\end{itemize}

\subsection{Zielgruppe}
\begin{itemize}
\item Autofahrer
\item Beifahrer
\item Personen, welche Routen für diese zuteilen bzw. erstellen
\end{itemize}

\subsection{Betriebsbedingungen}
\begin{itemize}
\item Zuhause oder in Büroumgebungen
\end{itemize}

\section{Produktumgebung}
%Kevin

\subsection{Software}

\begin{itemize}
\item Betriebssystem: Linux, Windows, andere Betriebssystem mit Java $\geq$ 7
\item Java-Laufzeitumgebung: Version 7 oder neuer
\end{itemize}

\subsection{Hardware}

\begin{itemize}
\item Ein Standard-PC mit angeschlossener Maus und Tastatur
\item Dieser sollte über einen Farbbildschirm mit der Auflösung von mindestens 1024$\times$768 Pixeln verfügen
\item Es muss genügend Arbeitsspeicher (mindestens 2 Gigabyte) und Festplattenkapazität (mindestens 2 Gigabyte freier Speicher) vorhanden sein
\item Es muss die oben genannte Software auf dem Computer lauffähig sein und bereits installiert und konfiguriert sein
\end{itemize}

\section{Funktionale Anforderungen}
% Betreuer: Weniger Text(gemacht)
% Felix
\subsection{Kernfunktionen}
\begin{description}
\oitem{F}{berechneNächstenPunktAufKante}
Berechnen des nächsten Punkts auf einer Kante zu Geokoordinaten auf der Karte.
\oitem{F}{berechneRoute}
Berechnen einer Route zwischen zwei Punkten auf je einer Straße des Straßengraphs.
\end{description}
\subsection{Benutzerschnittstelle}
\begin{description}
\oitem{F}{berechneKachel}
Berechnen einer Bildkachel aus lokalen Kartendaten zu gegebener Detailstufe und gegebenem Ausschnitt.
\oitem{F}{bestimmeKnotenInAusschnitt}
Bestimmen aller Knoten aus einem gegebenen Kartenausschnitt. (Benötigt für \ref{F:berechneKachel})
\oitem{F}{fetcheOnlineKachel}
Aufbauen einer HTTP-Verbindung und anfordern von Online-Kacheln.
\oitem{F}{zeichneKachel}
Darstellen von Kartenkacheln der entsprechenden Zoomstufe an der richtigen Stelle im GUI.
\oitem{F}{bestimmeKoordinatenAusPunkt}
Bestimmen der Geokoordinaten zu einem gegebenem Pixel auf der angezeigten Karte.
\oitem{F}{eingabeZuAusschnitt}
Umrechnung von Benutzereingaben auf der Karte in neuen Kartenausschnitt.
\oitem{F}{zeigeRoute}
Anzeigen einer berechneten Route (\ref{F:berechneRoute}) auf den dargestellten Kacheln (\ref{F:zeichneKachel}). Beispiel in
Abbildung \ref{fig:mockupscreenshot}
\end{description}
\subsection{Wegbeschreibung}
\begin{description}
\oitem{F}{bestimmeAbbiegeAusRoute}
Bestimmen aller Abbiegevorgänge einer Route.
\oitem{F}{klassifiziereAbbiege}
Klassifiziere die Abbiegevorgänge (rechts, links, scharf rechts, scharf links, geradeaus, halblinks, halbrechts, halblinks, rechts halten, links halten) und Abbiegevorqänge im Kreisverkehr (Ausfahrten zählen).
\oitem{F}{bestimmeAnweisungAusAbbiege}
Bestimmen der Anweisung aus einem Abbiegevorgang.
\oitem{F}{beschreibungZuHtml}
Erzeugen einer HTML-Datei aus einer Wegbeschreibung.
\oitem{F}{routeZuNavi}
Erzeugen einer Datei mit Geokoordinaten für die Benutzung in Navigationsgeräten.
\end{description}
\subsection{Profilverwaltung}
\begin{description}
\oitem{F}{neuesProfil} Neues Profil anlegen
\oitem{F}{löscheProfil} Vorhandenes Profil löschen
\oitem{F}{ändereProfil} Vorhandenes Profil ändern
\end{description}

\section{Produktdaten}
% Betreuer: Darstellung und Berechnung nur aus eigenen Daten. Steht noch nicht da.
%Anastasia
\subsection{Profildaten}
\begin{description}
\oitem{D}{fahrzeugTyp}
Es wird der Typ des Fahrzeugs (PKW, LKW, Bus) gespechert.
\oitem{D}{pkwDaten}
Für ein Fahrzeug vom Typ  PKW wird die vom Fahrer angegebene Durchschnittsgeschwindigkeit gespeichert.
\oitem{D}{lkwDaten}
Für ein Fahrzeug vom Typ  LKW werden folgende Daten gespeichert:\\
Länge, Breite, Höhe und Gewicht des Autos, maximale auf einer Autobahn erlaubte Geschwindigkeit.
\oitem{D}{busDaten}
Für ein Fahrzeug vom Typ Bus werden folgende Daten gespeichert:\\
Höhe und maximale auf einer Autobahn erlaubte Geschwindigkeit.
\end{description}

\subsection{Kartendaten}
\begin{description}
\oitem{D}{osmDatei}
Es wird eine OSM-Datei gespeichert.
\oitem{D}{graph}
Für eine bestimmte OSM-Datei wie in \ref{D:osmDatei} und ein bestimmten Typ des Fahrzeugs wie in \ref{D:fahrzeugTyp} wird ein Graph gespeichert.
\oitem{D}{graphDaten} % Betreuer: aufsplitten
Für einen Graphen wie in \ref{D:graph} werden folgende Daten gespeichert:\\
Knoten und Kanten des Graphen, Gewichte der Kanten, zusätzliche Information über die Angehörigkeit einer Kante zum kürzesten Pfad in Form von Arc-Flags.
\oitem{D}{knoten}
Für jeden Knoten werden geographischen Koordinaten (Länge und Breite) und eine unikale Identifikationsnummer.
\end{description}

\subsection{Historydaten}
\begin{description}
\oitem{D}{history}
Für eine bestimmte OSM-Datei wie in \ref{D:osmDatei} und ein bestimmten Typ des Fahrzeugs wie in \ref{D:fahrzeugTyp} wird eine bestimmte Anzahl von Ergebnissen der vorherigen Anfragen des Benutzers gespeichert, was als History bezeichnet wird.  
\end{description}
\section{Systemmodell} % Betreuer: Paketdiagramm
\includegraphics[width=\linewidth]{Systemmodell1}
%Anastasia
\section{Nichtfunktionale Anforderungen und Qualitätsziele}

\begin{description}
\oitem{NF}{routeFünfSek} \ref{F:berechneRoute} (Berechnung der Route) ist schneller als fünf Sekunden.
\oitem{NF}{knotenHalbeSek} \ref{F:berechneNächstenPunktAufKante} benötigt höchstens 500 ms.
\oitem{NF}{knotenWaysAnzahl} Die Software muss Daten mit mindestens x Knoten und y Ways verarbeiten können.
\end{description}
% Dominic
\section{Benutzeroberfläche}
% Betreuer: weniger Klicks? Startknopf.
% Betreuer: Elemente durchnummerieren. Einzeln erklären.
\begin{figure}[h]
\centering
\includegraphics[width=0.7\linewidth]{mockup_screenshot}
\caption{Erster Entwurf}
\label{fig:mockupscreenshot}
\end{figure}
% Fabian

Wir haben mehrere Profile und mehrere Karten.\\
Es ist immer eine Kombination aus Profil und Karte ausgewählt egal ob diese Vorberechnet ist oder nicht..

Wenn sie nicht vorberechnet ist (wird die Karte grau /deaktiviert/etc.), dann kann angeklickt werden (großer freundlicher Knopf), dass sie vorberechnet wird.
Wenn sie vorberechnet ist, dann kann angeklickt werden, dass die Vorberechnung gelöscht wird.

Die Routenberechnung startet, wenn Start oder Ziel geändert werden und danach beide vorhanden sind.

\section{Globale Testfälle und Szenarien}
% Betreuer: Vorbedinung, Aktionen, Nachbedingung
\subsection{Funktionssequenzen}
\begin{description}
\item\testfall
    {Karte ist sichtbar, Profil ist vorberechnet (für die aktuelle Karte)}
    {Auswählen des Start- bzw. Zielpunkts}
    {Start- bzw. Zielpunkt ist ausgewählt}
\item\testfall
    {Karte ist sichtbar, Profil ist vorberechnet(für die aktuelle Karte),  Start- bzw. Zielpunkt ist ausgewählt}
    {Auswählen des  Ziel- bzw. Startpunkts}
    {Route ist berechnet und wird angezeigt}
\item\testfall
    {Karte ist sichtbar}
    {Ziehen der Karte}
    {Die Kartenposititon hat sich entsprechend verändert}
\item\testfall
    {Karte ist sichtbar}
    {Zoomen}
    {Die Karte ist jetzt größer}
\item\testfall
    {Programm ist gestartet, aktuelles Profil ist nicht vorberechnet(für die aktuelle Karte)}
    {Vorberechnung auswählen, lange Warten}
    {aktuelles Profil ist vorberechnet(für die aktuelle Karte)}
\item\testfall
    {Programm ist gestartet}
    {neues Profil anlegen}
    {das neue Profil ist ausgewählt, das neue Profil ist nicht vorberechnet(für die aktuelle Karte)}
\item\testfall
    {Programm ist gestartet}
    {ein Profil wird ausgewählt}
    {das Profil ist ausgewählt}
\item\testfall
    {Programm ist gestartet}
    {ein Profil wird gelöscht}
    {das Profil ist nicht mehr vorhanden}
\item\testfall
    {Programm ist gestartet}
    {eine neue Karte wird importiert}
    {die neue Karte ist aktuell, kein Profil ist für die aktuelle Karte vorberechnet}
\item\testfall
    {Programm ist gestartet, eine Karte ist ausgewählt}
    {eine andere Karte wird ausgewählt}
    {die andere Karte ist ausgewählt}
\item\testfall
    {Programm ist gestartet, eine Route ist berechnet}
    {die Route wird gespeichert}
    {die Route ist abgespeichert}
\item\testfall
    {Programm ist gestartet, eine Route ist abgespeichert}
    {die abgespeicherte Route wird geladen}
    {die abgespeicherte Route wird wieder angezeigt}

\end{description}
\subsection{Datenkonsistenzen}

\begin{description}
\oitem{TD}{datenProfilVorberechnung}
Eine Route kann nur berechnet werden, wenn \gls{kartendaten} vorliegen, ein Profil ausgewählt wurde und die \gls{vorberechnung} für dieses \gls{profil} für diese Karte abgeschlossen ist.
\oitem{TD}{gültigOsm}
\gls{kartendaten} müssen als gültige osm-Datei vorliegen.
\oitem{TD}{karteProfilGelöscht}
Eine gespeicherte Route kann nicht angezeigt werden, wenn die zugehörigen Kartendaten und das verwendete \gls{profil} nicht mehr vorhanden sind.
\oitem{TD}{löscheNichtStandard}
Das \gls{standardprofil} kann nicht gelöscht werden.
\oitem{TD}{profil}
Es ist ein \gls{profil} ausgewählt.
\oitem{TD}{karte}
Es ist eine Karte ausgewählt.
\end{description}
% Fabian
\section{Entwicklungsumgebung}
\begin{description}
\item[Teamkommunikation] E-Mail-Verteiler
\item[Dokumentation] \LaTeX{}
  \item[UML-Planungswerkzeug] UMLet
  \item[IDE] Eclipse
  \item[Qualitätssicherung] JUnit, CodeCover
\item[Versionskontrolle] Git
\end{description}
% Kevin

\glsaddallunused
\printglossary[type=main, title={Glossar}, toctitle={Glossar}]

\end{document}
