% das Papierformat zuerst
\documentclass[a4paper, 11pt]{article}
\usepackage[utf8]{inputenc}
\usepackage[T1]{fontenc}
\usepackage[ngerman]{babel}

 
\title{Pflichtenheft}
%\author{}
%\date{}

% hier beginnt das Dokument
\begin{document}
\shorthandoff{"}

\maketitle
\newpage
\tableofcontents
\newpage


\section{Einleitung}

\section{Zielbestimmung}
\subsection{Musskriterien}

\subsection{Wunschkriterien}

\subsection{Abgerenzungskriterien}


\section{Produkteinsatz}

\section{Produktumgebung}

\section{Funktionale Anforderungen}

\section{Produktdaten}

\section{Systemmodell}

\section{Nichtfunktionale Anforderungen}

\section{Benutzeroberfläche}

\section{Qualitätsziele}

\section{Globale Testfälle}

\section{Entwicklungsumgebung}

\section{Glossar}

\end{document}
