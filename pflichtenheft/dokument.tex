% das Papierformat zuerst
\documentclass[a4paper, 11pt]{article}
\usepackage[utf8]{inputenc}
\usepackage[T1]{fontenc}
\usepackage[ngerman]{babel}
\usepackage{hyperref} % clickable refs
\usepackage{graphicx}
\usepackage[toc, numberedsection, style=altlist]{glossaries}
\makeglossary

 
\title{Pflichtenheft}
%\author{}
%\date{}

%Hack for referencing labels
\makeatletter
\def\namedlabel#1#2{\begingroup
    #2%
    \def\@currentlabel{#2}%
    \phantomsection\label{#1}\endgroup
}
\makeatother
% End: Hack for referencing labels

% Glossar: alle Einträge, aber ohne extra Referenzen
% http://tex.stackexchange.com/questions/115635/glossaries-suppress-pages-when-using-glsaddall
\newcommand*{\glsgobblenumber}[1]{}
\makeatletter
\newcommand*{\glsaddnp}[2][]{
  \glsdoifexists{#2}{
    \def\@glsnumberformat{glsgobblenumber}
    \edef\@gls@counter{\csname glo@#2@counter\endcsname}
    \setkeys{glossadd}{#1}
    \@gls@saveentrycounter
    \@do@wrglossary{#2}
  }
}
\newcommand{\glsaddallunused}[1][]{
  \edef\@glo@type{\@glo@types}
  \setkeys{glossadd}{#1}
  \forallglsentries[\@glo@type]{\@glo@entry}{
    \ifglsused{\@glo@entry}{}{
      \glsaddnp[#1]{\@glo@entry}}}
}
\makeatother

\renewcommand{\glsnamefont}[1]{\mdseries #1} % glossary entries shouldn’t be bold

% Glossar

% So sieht ein Glossar-Eintrag aus:
%
%\newglossaryentry{dijkstra}{
%  name={Dijkstra’s Algorithmus},
%  description={ein Algorithmus, um den optimalen Pfad in einem gerichteten Graphen zu finden}
%}
%\newglossaryentry{arc}{
%  name={Arc-Flags},
%  description={eine Technik, um Routenberechnung zu beschleunigen},
%  see=[siehe auch]{dijkstra}
%}
%
% Und so kann er im Dokument verwendet werden:
%
% lorem ipsum dolor sit \gls{arc}, consectetur
%
% End: Glossar

% usage: \counteditem{prefix}{refName} -> item '/prefixXX/' with label 'prefix:refName' (where XX is counted in increments of 10)
\makeatletter
\newcommand{\oitem}[2]{
  % define the counter
  \@ifundefined{c@oitem#1}{\newcounter{oitem#1}}{} % initialized at 0
  \addtocounter{oitem#1}{10}
  \item[\namedlabel{#1:#2}{/#1\arabic{oitem#1}/}]
}
\makeatother

\newcommand{\testfall}[3]{
  \begin{description}
    \item[Vorbedingungen] #1
    \item[Ablauf] #2
    \item[Erwartetes Ergebnis] #3
  \end{description}
}

% Betreuer: Rand reduzieren.
% Betreuer: Wichtig, viel Kriterien, GUI, Testfälle
% Betreuer: Präzise!!!
% hier beginnt das Dokument
\begin{document}
\shorthandoff{"}

% place a symbol before clickable links
% this has to come *after* \begin{document} because hyperref installs a \AtBeginDocument hook that updates the ref command.
\newcommand{\refsymbol}[0]{\scalebox{0.5}{$\nearrow$}}
\let\oldref\ref
\renewcommand{\ref}[1]{\refsymbol\oldref{#1}}
\let\oldgls\gls
\renewcommand{\gls}[1]{\refsymbol\oldgls{#1}}

\maketitle
\newpage
\tableofcontents
\newpage
% Betreuer: Berechnung von Punkt auf Kante zu Punkt auf Kante.  (KdBaum über Kanten)
% Betreuer: Profile exakt welcher Typ
\section{Einleitung}
\newglossaryentry{route}{
  name={Route},
  description={Ein Weg zwischen zwei Punkten}
}
Das Produkt ist an Autofahrer und/oder deren Mitfahrer gerichtet; es soll ihnen die Planung einer Reise erleichtern, indem es ihnen die zeit effizienteste \gls{route} zwischen zwei eingegebenen Punkten berechnet und anzeigt.
Dabei wird besonders darauf Wert gelegt, dass die gewählte Route auch gefahren werden kann. Dazu werden verschiedene Beschränkungen (Abbiegebeschränkungen, Beschränkungen aufgrund der Größe oder des Gewichts des Fahrzeugs, ...) bei der Berechnung der Route berücksichtigt.

Das Produkt soll auf jedem handelsüblichen Desktop-Computer, wie ihn die Mehrheit der Zielgruppe besitzt, lauffähig sein. Die Kartendaten stammen aus dem OpenStreetMap-Projekt.
%Lucas

% Betreuer: Abbiegungbeschränkung, "Route berechnen", 
\section{Zielbestimmung}
%Lucas

\subsection{Musskriterien}
% Betreuer: Automatisch Nummerieren
% Betreuer: Ausführlicher.
\begin{itemize}
\item Karte anzeigen
\item Karte verschieben
\item Karte zoomen
\item Start- und Zielpunkt können ausgewählt werden. % Rechtsklick
\item Start- und Zielpunkt auf der Karte anzeigen
\item Route zwischen Start- und Zielpunkt anfragen
\item Berechnen der zeit effizientesten Route zwischen Start- und Zielpunkt
\item Anzeigen der zeit effizientesten Route  zwischen Start- und Zielpunkt auf der Karte
\end{itemize}
\subsection{Wunschkriterien}
\begin{itemize}
\newglossaryentry{arc}{
 name={Arc-Flags},
 description={eine Technik, um Routenberechnung zu beschleunigen}
}
\newglossaryentry{beschleunigung}{
 name={Beschleunigung},
 description={Ermöglicht eine schnellere Berechnung der angefragten Route},
see=[siehe auch]{arc}
}
\item \gls{beschleunigung} durch \gls{arc} bei der Berechnung der Route
\item Umsetzen möglichst vieler Beschränkungen (Staßentyp, Abbiegebeschränkung, 
Maximalgewicht, Einbahnstraße) trotz Beschleunigung. Profile für die Vorberechnung. % Aufsplitten, ungenau.

\item Wegbeschreibung anzeigen. (Bei Kreisverkehren: Ausfahrten zählen)
\item Berechnungsanfragen speichern. History/Favoriten
\item Export der Route.

\item Optional OSM-Kacheln
% Betreuer: Profile

\end{itemize}
\subsection{Abgerenzungskriterien}

\begin{itemize}
\item Keine textuelle Suche möglich
\item Kein Routenplaner für Fahrrad, Fußgänger, Bahnfahrer oder etc.
\item Keine Mobile App.
\item Keine Webanwendung
\item Keine Alternativroute sofort anzeigen

\end{itemize}

\section{Produkteinsatz}
%Kevin

Das Produkt soll Autofahrern bei der Planung einer optimalen Fahrstrecke helfen.

\subsection{Anwendungsbereiche}
\begin{itemize}
\item Routenplanung
\end{itemize}

\subsection{Zielgruppe}
\begin{itemize}
\item Autofahrer
\item Beifahrer
\item Personen, welche Routen für diese zuteilen bzw. erstellen
\end{itemize}

\subsection{Betriebsbedingungen}
\begin{itemize}
\item Zuhause oder in Büroumgebungen
\end{itemize}

\section{Produktumgebung}
%Kevin

\subsection{Software}

\begin{itemize}
\item Betriebssystem: Linux, Windows, andere mit Java $\geq$ 6
\item Java-Laufzeitumgebung: Version 6 oder neuer
\end{itemize}

\subsection{Hardware}

\begin{itemize}
\item Ein Standard-PC mit angeschlossener Maus und Tastatur
\item Dieser sollte über einen Farbbildschirm mit der Auflösung von mindestens 1024$\times$768 Pixeln verfügen
\item Es muss genügend Arbeitsspeicher (mindestens 2 Gigabyte) und Festplattenkapazität (mindestens 2 Gigabyte freier Speicher) vorhanden sein
\item Es muss die oben genannte Software auf dem Computer lauffähig sein und bereits installiert und konfiguriert sein
\end{itemize}

\section{Funktionale Anforderungen}
% Betreuer: Weniger Text
% Felix
\subsection{Kernfunktionen}
\begin{description}
\oitem{F}{berechneNächstenPunktAufKante}
Berechnen des nächsten Punkt auf einer Kante zu Geokoordinaten auf der Karte.
\oitem{F}{berechneRoute}
Berechnen einer Route zwischen zwei Punten auf je einer Straße des Straßengraphs.
\end{description}
\subsection{Benutzerschnittstelle}
\begin{description}
\oitem{F}{berechneKachel}
Berechnen einer Bildkachel aus lokalen Kartendaten zu gegebener Detailstufe und gegebenem Ausschnitt.
\oitem{F}{bestimmeKnotenInAusschnitt}
Bestimmen aller Knoten aus einem gegebenen Kartenausschnitt. (Benötigt für \ref{F:berechneKachel})
\oitem{F}{fetcheOnlineKachel}
Aufbauen einer HTTP-Verbindung und anfordern von Online-Kacheln.
\oitem{F}{zeichneKachel}
Darstellen von Kartenkacheln der entsprechenden Zoomstufe an der richtigen Stelle im GUI angezeigt werden.
\oitem{F}{bestimmeKoordinatenAusPunkt}
Bestimmen der Geokoordinaten zu einem gegebenem Pixel auf der angezeigten Karte.
\oitem{F}{eingabeZuAusschnitt}
Umrechnung von Benutzereingaben auf der Karte in neuen Kartenausschnitt.
\oitem{F}{zeigeRoute}
Anzeigen einer berechneter Route auf den dargestellten Kacheln (\ref{F:zeichneKachel}). Beispiel in
Abbildung \ref{fig:mockupscreenshot}
\end{description}
\subsection{Wegbeschreibung}
\begin{description}
\oitem{F}{bestimmeAbbiegeAusRoute}
Bestimmen aller Abbiegevorgänge einer Route.
\oitem{F}{klassifiziereAbbiege}
Klassifiziere die Abbiegevorgänge (rechts, links, scharf rechts, scharf links, geradeaus, halblinks, halbrechts, halblinks, rechts halten, links halten) und Abbiegevorqänge im Kreisverkehr (Ausfahrten zählen).
\oitem{F}{bestimmeAnweisungAusAbbiege}
Bestimmen der Anweisung aus einem Abbiegevorgang.
\oitem{F}{beschreibungZuHtml}
Erzeugen einer HTML-Datei aus einer Wegbeschreibung.
\oitem{F}{routeZuNavi}
Erzeugen einer Datei mit Geokoordinaten für die Benutzung in Navigationsgeräten.
\end{description}
\subsection{Profilverwaltung}
\begin{description}
\oitem{F}{neuesProfil} Neues Profil anlegen
\oitem{F}{löscheProfil} Vorhandenes Profil löschen
\oitem{F}{ändereProfil} Vorhandenes Profil ändern
\end{description}

\section{Produktdaten}
% Betreuer: Darstellung und Berechnung nur aus eigenen Daten. Steht noch nicht da.
%Anastasia
\subsection{Profildaten}
\begin{description}
\oitem{D}{fahrzeugTyp}
Es wird der Typ des Fahrzeugs (PKW, LKW, Bus) gespechert.
\oitem{D}{pkwDaten}
Für ein Fahrzeug vom Typ  PKW wird die vom Fahrer angegebene Durchschnittsgeschwindigkeit gespeichert.
\oitem{D}{lkwDaten}
Für ein Fahrzeug vom Typ  LKW werden folgende Daten gespeichert:\\
Länge, Breite, Höhe und Gewicht des Autos, maximale auf einer Autobahn erlaubte Geschwindigkeit.
\oitem{D}{busDaten}
Für ein Fahrzeug vom Typ Bus werden folgende Daten gespeichert:\\
Höhe und maximale auf einer Autobahn erlaubte Geschwindigkeit.
\end{description}

\subsection{Kartendaten}
\begin{description}
\oitem{D}{osmDatei}
Es wird eine OSM-Datei gespeichert.
\oitem{D}{graph}
Für eine bestimmte OSM-Datei wie in \ref{D:osmDatei} und ein bestimmten Typ des Fahrzeugs wie in \ref{D:fahrzeugTyp} wird ein Graph gespeichert.
\oitem{D}{graphDaten} % Betreuer: aufsplitten
Für einen Graphen wie in \ref{D:graph} werden folgende Daten gespeichert:\\
Knoten und Kanten des Graphen, Gewichte der Kanten, zusätzliche Information über die Angehörigkeit einer Kante zum kürzesten Pfad in Form von Arc-Flags.
\oitem{D}{knoten}
Für jeden Knoten werden geographischen Koordinaten (Länge und Breite) und eine unikale Identifikationsnummer.
\end{description}

\subsection{Historydaten}
\begin{description}
\oitem{D}{history}
Für eine bestimmte OSM-Datei wie in \ref{D:osmDatei} und ein bestimmten Typ des Fahrzeugs wie in \ref{D:fahrzeugTyp} wird eine bestimmte Anzahl von Ergebnissen der vorherigen Anfragen des Benutzers gespeichert, was als History bezeichnet wird.  
\end{description}
\section{Systemmodell} % Betreuer: Paketdiagramm
\includegraphics[width=\linewidth]{Systemmodell1}
%Anastasia
\section{Nichtfunktionale Anforderungen}

\begin{description}
\oitem{NF}{routeFünfSek} \ref{F:berechneRoute} (Berechnung der Route) ist schneller als fünf Sekunden.
\oitem{NF}{knotenHalbeSek} \ref{F:berechneNächstenPunktAufKante} benötigt höchstens 500 ms.
\oitem{NF}{knotenWaysAnzahl} Die Software muss Daten mit mindestens x Knoten und y Ways verarbeiten können.
\end{description}
% Dominic
\section{Benutzeroberfläche}
% Betreuer: weniger Klicks? Startknopf.
% Betreuer: Elemente durchnummerieren. Einzeln erklären.
\begin{figure}
\centering
\includegraphics[width=0.7\linewidth]{mockup_screenshot}
\caption{Erster Entwurf}
\label{fig:mockupscreenshot}
\end{figure}
% Fabian
\section{Qualitätsziele}
% Dominic
\section{Globale Testfälle und Szenarien}
% Betreuer: Vorbedinung, Aktionen, Nachbedingung
\subsection{Funktionssequenzen}
\newglossaryentry{profil}{
  name={Profil},
  description={Enthält für die Routenplanung relevante Informationen, z.B. die Höhe und das Gewicht des Fahrzeugs}
}
\begin{description}
\item\testfall
    {Programm ist gestartet, Karte ist sichtbar}
    {Rechtsklick auf Karte}
    {Kontextmenü erscheint}
\oitem{TF}{startZielRechne}
Start auswählen, Ziel auswählen, Route berechnen, Route auf Karte anzeigen
\oitem{TF}{bewegeZoom}
Karte bewegen, heranzoomen, Karte bewegen, herauszoomen
\oitem{TF}{profilErstellen}
Profilerstellung, Vorberechnung, Route berechnen, Profil löschen
\oitem{TF}{profilWechseln}
Route berechnen, \gls{profil} wechseln, Route berechnen
\oitem{TF}{ladenBerechnen}
Neue Kartendaten laden, Vorberechnung, Route berechnen
\oitem{TF}{kartenWechsel}
Route berechnen, Karte wechseln, Route berechnen
\oitem{TF}{routeLaden}
Route berechnen, Route speichern, andere Route berechnen, Kartenausschnitt verändern, gespeicherte Route laden
\end{description}
\subsection{Datenkonsistenzen}
\newglossaryentry{kartendaten}{
  name={Kartendaten},
  description={Eine Datei im osm-Format, die eine Karte in Form von Knoten, Kanten und Relationen mit weiteren Informationen enthält}
}
\newglossaryentry{standardprofil}{
  name={Standardprofil},
  description={Ein vorinstalliertes Profil für gängige PKWs}
}
\newglossaryentry{vorberechnung}{
  name={Vorberechnung},
  description={Wandelt die Kartendaten in ein effizienteres Format um und fügt auf dem Profil basierende Informationen hinzu, die eine schnelle Routenberechnung ermöglichen. Muss für jede neue Karte/Profil-Kombination einmal ausgeführt werden}
}

\begin{description}
\oitem{TD}{datenProfilVorberechnung}
Eine Route kann nur berechnet werden, wenn \gls{kartendaten} vorliegen, ein Profil ausgewählt wurde und die \gls{vorberechnung} für dieses \gls{profil} für diese Karte abgeschlossen ist.
\oitem{TD}{gültigOsm}
\gls{kartendaten} müssen als gültige osm-Datei vorliegen.
\oitem{TD}{karteProfilGelöscht}
Eine gespeicherte Route kann nicht angezeigt werden, wenn die zugehörigen Kartendaten und das verwendete \gls{profil} nicht mehr vorhanden sind.
\oitem{TD}{löscheNichtStandard}
Das \gls{standardprofil} kann nicht gelöscht werden.
\end{description}
% Fabian
\section{Entwicklungsumgebung}
\begin{description}
\item[Teamkommunikation] E-Mail-Verteiler
\item[Dokumentation] \LaTeX{}
  \item[UML-Planungswerkzeug] UMLet
  \item[IDE] Eclipse
  \item[Qualitätssicherung] JUnit, CodeCover
\item[Versionskontrolle] Git
\end{description}
% Kevin

\glsaddallunused
\printglossary[type=main, title={Glossar}, toctitle={Glossar}]

\end{document}
